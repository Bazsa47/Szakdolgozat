\documentclass{thesis-ekf}
\usepackage[T1]{fontenc}
\PassOptionsToPackage{defaults=hu-min}{magyar.ldf}
\usepackage[magyar]{babel}
\usepackage{mathtools,amssymb,amsthm}
\footnotestyle{rule=fourth}

\newtheorem{tetel}{Tétel}[chapter]
\theoremstyle{definition}
\newtheorem{definicio}[tetel]{Definíció}
\theoremstyle{remark}
\newtheorem{megjegyzes}[tetel]{Megjegyzés}

\begin{document}
\institute{Matematikai és Informatikai Intézet}
\title{Többszemélyes játék mesterséges intelligenciával}
\author{Farkas Balázs Alex\\Programtervező Informatikus BSc}
\supervisor{Dr. Kovásznai Gergely\\egyetemi docens}
\city{Eger}
\date{2021}
\maketitle
\tableofcontents

\chapter*{Bevezetés}
miről szól a szakdoga, alap probléma ,technológiák, cél,(meddig jutottunk)
\chapter{Irodalmi áttekintés} - alternatívák, miért a unity?
\chapter{Játékmenet}
\chapter{Többszemélyes játék}
\chapter{Mesterséges intelligencia}
\chapter{összegzés} - bevezetés múlt időben, elért és nem elért célok, konklúziók, jövő beli lehetőségek

\begin{thebibliography}{2}

\end{thebibliography}
\end{document}